\section{Allgemeine Punkte} 
\begin{frame}
\frametitle{Code} 
\begin{itemize}
\item veröffentlichbarer Code
\item lesbarer Code
\item wartbarer Code
\item Trennung von Technik und Fachlichkeit
\end{itemize}
\begin{block}{Ziel}
  Grundsätzlich soll Code erzeugt werden der auf github veröffnetlicht werden kann (nicht muss)
  \end{block}
\end{frame}
\begin{frame}
\frametitle{Sprache} 
\begin{itemize}
\item Entwickliung in Englisch
\item Kommentare in Englisch
\item Dokumentation ausserhalb in deutsch, englisch oder mandarin
\end{itemize}
\begin{block}{Ziel}
  Form follows function - Pragmatische Auswahl der Sprache nach Bedarf, Anwendung und Ziel
  \end{block}
\end{frame}
\begin{frame}
    \frametitle{Steile Thesen} 
    \begin{itemize}
    \item Jede Iteration im Produkt muss das Produkt besser machen
    \item Wenn Excel Teil deines Deployments ist solltest du deine Berufswahl ueberdenken
    \item Wenn die Kommandozeile / Rumgefrickel Teil deines Deployments ist solltest du dein Deployment ueberdenken
    \item Wenn dein Companero deine Vorgehensweise nicht versteht solltest du deine Vorgehensweise ueberdenken
    \item Wenn dein Abnehmer dein Produkt nicht versteht liegt es nicht daran dass dieser kein Informatiker ist.
    \item git clone sollte ausreichen um eine luaffaehige Entwicklungsumgebung zu haben (vorrausgesetyt du sitzt an einem Standardentwicklungsrechner)
    \end{itemize}
    
    \end{frame}