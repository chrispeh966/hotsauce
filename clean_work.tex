% Dieser Text ist urheberrechtlich gesch\"utzt
% Er stellt einen Auszug eines von mir erstellten Referates da
% und darf nicht gewerblich genutzt werden
% die private bzw. Studiums bezogen Nutzung ist frei
% April 2011
% Autor: Sascha Frank 
% Universit\"at Freiburg 
% www.informatik.uni-freiburg.de/~frank/
% frank < was da sonst immer steht > tf.uni-freiburg.de


\documentclass[hyperref={pdfpagelabels=false}]{beamer}
% Die Hyperref Option hyperref={pdfpagelabels=false} verhindert die Warnung:
% Package hyperref Warning: Option `pdfpagelabels' is turned off
% (hyperref)                because \thepage is undefined. 
% Hyperref stopped early 
%

\usepackage{lmodern}
% Das Paket lmodern erspart die folgenden Warnungen:
% LaTeX Font Warning: Font shape `OT1/cmss/m/n' in size <4> not available
% (Font)              size <5> substituted on input line 22.
% LaTeX Font Warning: Size substitutions with differences
% (Font)              up to 1.0pt have occurred.
%

% Wenn \titel{\ldots} \author{\ldots} erst nach \begin{document} kommen,
% kommt folgende Warnung:
% Package hyperref Warning: Option `pdfauthor' has already been used,
% (hyperref) ... 
% Daher steht es hier vor \begin{document}

\title{Entwicklungsguidelines}   
\author{nope} 
\date{\today} 

% zusaetzlich ist das usepackage{beamerthemeshadow} eingebunden 
\usepackage{beamerthemeshadow}

%  \beamersetuncovermixins{\opaqueness<1>{25}}{\opaqueness<2->{15}}
%  sorgt dafuer das die Elemente die erst noch (zukuenftig) kommen 
%  nur schwach angedeutet erscheinen 
\beamersetuncovermixins{\opaqueness<1>{25}}{\opaqueness<2->{15}}
% klappt auch bei Tabellen, wenn teTeX verwendet wird\ldots
\begin{document}


\begin{frame}
\titlepage
\end{frame} 

\begin{frame}
\frametitle{Inhaltsverzeichnis}
\tableofcontents
\end{frame} 


\section{Allgemeine Punkte} 
\begin{frame}
\frametitle{Code} 
\begin{itemize}
\item veröffentlichbarer Code
\item lesbarer Code
\item wartbarer Code
  \item Trennung von Technik und Fachlichkeit
\end{itemize}
\begin{block}{Ziel}
  Grundsätzlich soll Code erzeugt werden der auf github veröffnetlicht werden kann (nicht muss)
  \end{block}
\end{frame}
\begin{frame}
\frametitle{Sprache} 
\begin{itemize}
\item Entwickliung in Englisch
\item Kommentare in Englisch
\item Dokumentation ausserhalb in deutsch, englisch oder mandarin
\end{itemize}
\begin{block}{Ziel}
  Form follows function - Pragmatische Auswahl der Sprache nach Bedarf, Anwendung und
  \end{block}
\end{frame}
\section{Modulstruktur}
\begin{frame}
  \frametitle{Nomenklatur}
  \begin{itemize}
  \item artifactId == Name des Moduls
  \item groupdid == de.Firmenname.bereichsname.projekt
    \item package name == groupdid.artifactid:Versionsnummer
  \end{itemize}
  \begin{block}{Ziel}
    Einheitliche Namensstrukturen, 
    \end{block}
\end{frame}
\begin{frame}
  \frametitle{Generelle Punkte}
  \begin{itemize}
  \item Multimodul vor Multiprojekt
  \item parent pom mit wählbaren plugins
  \item Querabhängigkeiten nie über snapshots
  \end{itemize}
  \begin{block}{Ziel}
    Keine Works on my machine Projekte - Einheitliche Strukturen. Einheitliche Arbeitsumgebung (TODO besseres Wortwahl)
    \end{block}
\end{frame}
\section{Codestyle}
\begin{frame}
  \frametitle{Formatter}
  \begin{itemize}
  \item Kein Eigenlösungen
  \item google codestyle als Basis
  \item ausschliesslich google formatter verwenden
  \end{itemize}
  \begin{block}{Prämisse}
    Jeder kann machen was er will .... Der Main branch muss konform zum google code style sein
    \end{block}
\end{frame}
6


\begin{frame} 
Denn ohne Titel fehlt ihnen was
\end{frame}

\end{document}
