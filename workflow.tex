\section{Workflow}
\begin{frame}
    \frametitle{Kollaboration}
    \begin{itemize}
        \item Ein Projekt sollte immer mindestens eine Kopilotin / einen Kopiloten haben
        \item Workflows duerfen kein Hindernis sein / Pragmatismus vor Dogmatismus
        \item Entwickelnde sollten nicht gleichzeitig Maintainende des eigenen Codes sein
    \end{itemize}
    \begin{block}{Ziel}
        Steile These: Wenn deine Companera dein Deployment / Code oder Projektstruktur nicht schnell versteht liegt es an Dir und nicht an deinem Companero
    \end{block}
\end{frame}
\begin{frame}
    \frametitle{Dokumentation}
    \begin{itemize}
        \item Form follows function
        \item Technische Anforderungen und Vorgehensweisen fuer das Deployment / Entwicklungsumgebung immer in die README
        \item Technische Dokumentationen immer im git und versionierbar (.md o.ae.)
        \item Benutzerdokumentation immer im git und versionierbar, jedoch exportierbar fuer confluence
    \end{itemize}
    
\end{frame}
\begin{frame}
    \frametitle{Branches}
    \begin{itemize}
    \item Main-branch heisst main
    \item Entwicklungsbranches sind (kurze) featurebranches
    \item Merge in main erfolgt immer ueber pull-request
    \item In der Regel sollte ein Code-review stattfinden
    \item Featurebranchname: $<JIRA-TicketNo>\_<sprechende\_Bezeichnung>$
    \end{itemize}
    \begin{block}{Ziel}
      Im main-branch sollte immer lauffaehiger, getesteter und deployfahiger Code liegen
      \end{block}
  \end{frame}

\begin{frame}
    \frametitle{git}
    \begin{itemize}
    \item squash commits bei merge
    \item Grundsaetzlich nur fast-forward merges
    \item Ansonsten: $ main  \rightarrow feature \rightarrow main $
    \item rebase $\rightarrow$ Gehe in das Gefängnis. Begib Dich direkt dorthin. Gehe nicht über Los. Ziehe nicht EUR 1000 ein 
    \end{itemize}
    \begin{block}{Ziel}
      Uebersichtliche und nachvollziehbare Entwicklungszyklen sowie git als single point of truth
      \end{block}
\end{frame}

\begin{frame}
    \frametitle{Testabdeckung}
    \begin{itemize}
    \item Grundsaetzlich sollte neuer Code mit Tests abgedeckt werden
    \item Auftretende Bugs sollten mit Tests geprueft werden
    \item Eine hohe Coverage garantiert keinen guten Code
    \item Gute Tests ermoeglichen bessere Produkte
    \end{itemize}
    \begin{block}{Ziel}
        Tests sollten die grundsaetzliche Funktionalitaeten pruefen und sicherstellen dass eine Software nach einem neuen Release nicht schlechter geworden ist.
    \end{block}
\end{frame}


\begin{frame}
    \frametitle{CI}
    \begin{itemize}
      \item Verwendung der Basis-CI
      \item Kein voodoo in der CI (dwyce - do what your customer expect)
      \item Feature-branch pushes sollen KEINE Artefakte o.ae. erzeugen
      \item Ein push auf main muss ein Artefakt erzeugen (release)
    \end{itemize}
    \begin{block}{Ziel}
        Die CI stellt sicher dass getesteter und lauffaehiger Code im repositry liegt. Zudem stellt Sie sicher dass Artefakte luaffaehig und versioniert abgelegt werden
    \end{block}
\end{frame}

\begin{frame}
    \frametitle{CI 2}
    \begin{itemize}
      \item Artifakte sollten nur als release abgelegt werden
      \item CI fuehrt tests durch z.B. spotless:check, pmd:check, sonarcube, jacoco \dots
      \item Builds sollen reproduzierbar sein
    \end{itemize}
    \begin{block}{Ziel}
        Einige Metriken der CI sollen auswertbar sein und Codeveraenderungen aussagekraeftig anzeigen koennen.        
    \end{block}
\end{frame}

