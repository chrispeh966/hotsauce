% Dieser Text ist urheberrechtlich gesch\"utzt
% Er stellt einen Auszug eines von mir erstellten Referates da
% und darf nicht gewerblich genutzt werden
% die private bzw. Studiums bezogen Nutzung ist frei
% April 2011
% Autor: Sascha Frank 
% Universit\"at Freiburg 
% www.informatik.uni-freiburg.de/~frank/
% frank < was da sonst immer steht > tf.uni-freiburg.de


\documentclass[hyperref={pdfpagelabels=false}]{beamer}
% Die Hyperref Option hyperref={pdfpagelabels=false} verhindert die Warnung:
% Package hyperref Warning: Option `pdfpagelabels' is turned off
% (hyperref)                because \thepage is undefined. 
% Hyperref stopped early 
%
\usetheme{Frankfurt}
\setbeamerfont*{frametitle}{size=\normalsize,series=\bfseries}
\setbeamertemplate{navigation symbols}{}


% Standard packages

\usepackage[german]{babel}
\usepackage[utf8]{inputenc}
\usepackage{times}
\usepackage[T1]{fontenc}

% Setup TikZ

\usepackage{tikz}
\usetikzlibrary{arrows}
\tikzstyle{block}=[draw opacity=0.7,line width=1.4cm]

\usepackage{lmodern}
% Das Paket lmodern erspart die folgenden Warnungen:
% LaTeX Font Warning: Font shape `OT1/cmss/m/n' in size <4> not available
% (Font)              size <5> substituted on input line 22.
% LaTeX Font Warning: Size substitutions with differences
% (Font)              up to 1.0pt have occurred.
%

% Wenn \titel{\ldots} \author{\ldots} erst nach \begin{document} kommen,
% kommt folgende Warnung:
% Package hyperref Warning: Option `pdfauthor' has already been used,
% (hyperref) ... 
% Daher steht es hier vor \begin{document}


\title{Entwicklungsguidelines}   
\author{nope} 
\date{\today} 

% zusaetzlich ist das usepackage{beamerthemeshadow} eingebunden 
\usepackage{beamerthemeshadow}

%  \beamersetuncovermixins{\opaqueness<1>{25}}{\opaqueness<2->{15}}
%  sorgt dafuer das die Elemente die erst noch (zukuenftig) kommen 
%  nur schwach angedeutet erscheinen 
\beamersetuncovermixins{\opaqueness<1>{25}}{\opaqueness<2->{15}}
% klappt auch bei Tabellen, wenn teTeX verwendet wird\ldots
\begin{document}


\begin{frame}
\titlepage
\end{frame} 

\begin{frame}
\frametitle{Inhaltsverzeichnis}
\tableofcontents
\end{frame} 


\section{Allgemeine Punkte} 
\begin{frame}
\frametitle{Code} 
\begin{itemize}
\item veröffentlichbarer Code
\item lesbarer Code
\item wartbarer Code
\item Trennung von Technik und Fachlichkeit
\end{itemize}
\begin{block}{Ziel}
  Grundsätzlich soll Code erzeugt werden der auf github veröffnetlicht werden kann (nicht muss)
  \end{block}
\end{frame}
\begin{frame}
\frametitle{Sprache} 
\begin{itemize}
\item Entwickliung in Englisch
\item Kommentare in Englisch
\item Dokumentation ausserhalb in deutsch, englisch oder mandarin
\end{itemize}
\begin{block}{Ziel}
  Form follows function - Pragmatische Auswahl der Sprache nach Bedarf, Anwendung und
  \end{block}
\end{frame}

  
\section{Codestyle}
\begin{frame}
  \frametitle{Formatter}
  \begin{itemize}
  \item google code style als Basis
  \item google formatter verwenden
  \item erreichbar über git-hooks bei commit und push
  \item erforderlich bei push in den main branch
  \end{itemize}
  \begin{block}{Prämisse}
    Eigene branches (feature oder bugfix) können eigene Formatierung haben. Der Main branch muss konform zum google code style sein.
    \end{block}
\end{frame}

\begin{frame}
  \frametitle{Aussagekräftige Funktions- und Methodennamen}
  \begin{itemize}
  \item it, has, does, can liefert einen \textbf{boolean} zurück
  \begin{Beispiel}
  \par public boolean \textbf{is}Active() 
  \par protected boolean \textbf{has}License()
    \end{Beispiel}
  \end{itemize}
\end{frame}

\begin{frame}
  \frametitle{Funktions- und Methodennamen}
  \begin{itemize}
  \item Funktion/Methode, die einen Wert zurückliefert, enthält entsprechendes Verb \par(get, fetch, compute, generate, create, calculate)
  \begin{Beispiel}
  \par public int \textbf{get}TotalCostFromStart()
  \par private int \textbf{calculate}EuclideanDistance(Node source, Node target)
  \par public String \textbf{generate}InviteCode(Role role, String identifier)
  \par private ValueGraph<Node, Integer> \textbf{create}Graph(List<List<Node> > nodes)
  \par private int \textbf{compute}Score(User user)
  \end{Beispiel}
  \end{itemize}

\end{frame}

\begin{frame}
  \frametitle{Funktions- und Methodennamen}
  \begin{itemize}
  \item nichtsprechende Verben liefern nichts zurück (void!) $\rightarrow$ Exception
  \begin{Beispiel}
  \par public void \textbf{check}Credentials())
  \end{Beispiel} 
  
  \item für checks bietet sich die Klasse Preconditions von guava an.
  \begin{Beispiel}
  \par checkArgument(value >= 0, ``input is negative: \%s", value);
  \end{Beispiel}
  \end{itemize}

\end{frame}

\begin{frame}
\frametitle{Funktions- und Methodennamen}
  \begin{itemize}
    \item sprechende Funktionssignaturen sind besser als Dokumentation
    \end{itemize}
  \begin{Beispiel}
  \par long \sout{getTimestamp()} $\rightarrow$ long getTimestamp\textbf{InMilliseconds}()
  \end{Beispiel}

\end{frame}
\section{Workflow}
\begin{frame}
    \frametitle{Branches}
    \begin{itemize}
    \item Main-branch heisst main
    \item Entwicklungsbranches sind (kurze) featurebranches
    \item Merge in main erfolgt immer ueber pull-request
    \item In der Regel sollte ein Code-review stattfinden
    \end{itemize}
    \begin{block}{Ziel}
      Im main-branch sollte immer lauffaehiger, getesteter und deployfahiger Code liegen
      \end{block}
  \end{frame}

  \begin{frame}
    \frametitle{git}
    \begin{itemize}
    \item squash commits bei merge
    \item Grundsaetzlich nur fast-forward merges
    \item Ansonsten: $ main  \rightarrow feature \rightarrow main $
    \item rebase $\rightarrow$ Gehe in das Gefängnis. Begib Dich direkt dorthin. Gehe nicht über Los. Ziehe nicht EUR 1000 ein 
    \end{itemize}
    \begin{block}{Ziel}
      Im main-branch sollte immer lauffaehiger, getesteter und deployfahiger Code liegen
      \end{block}
    \end{frame}




\end{document}
