% Dieser Text ist urheberrechtlich gesch\"utzt
% Er stellt einen Auszug eines von mir erstellten Referates da
% und darf nicht gewerblich genutzt werden
% die private bzw. Studiums bezogen Nutzung ist frei
% April 2011
% Autor: Sascha Frank 
% Universit\"at Freiburg 
% www.informatik.uni-freiburg.de/~frank/
% frank < was da sonst immer steht > tf.uni-freiburg.de


\documentclass[hyperref={pdfpagelabels=false}]{beamer}
% Die Hyperref Option hyperref={pdfpagelabels=false} verhindert die Warnung:
% Package hyperref Warning: Option `pdfpagelabels' is turned off
% (hyperref)                because \thepage is undefined. 
% Hyperref stopped early 
%
\usetheme{Frankfurt}
 \setbeamerfont*{frametitle}{size=\normalsize,series=\bfseries}
\setbeamertemplate{navigation symbols}{}


% Standard packages

\usepackage[german]{babel}
\usepackage[utf8]{inputenc}
\usepackage{times}
\usepackage[T1]{fontenc}
\usepackage{ulem}

% Setup TikZ

\usepackage{tikz}
\usetikzlibrary{arrows}
\tikzstyle{block}=[draw opacity=0.7,line width=1.4cm]

\usepackage{lmodern}
% Das Paket lmodern erspart die folgenden Warnungen:
% LaTeX Font Warning: Font shape `OT1/cmss/m/n' in size <4> not available
% (Font)              size <5> substituted on input line 22.
% LaTeX Font Warning: Size substitutions with differences
% (Font)              up to 1.0pt have occurred.
%

% Wenn \titel{\ldots} \author{\ldots} erst nach \begin{document} kommen,
% kommt folgende Warnung:
% Package hyperref Warning: Option `pdfauthor' has already been used,
% (hyperref) ... 
% Daher steht es hier vor \begin{document}


\title{Entwicklungsguidelines}   
\author{nope} 
\date{\today} 

% zusaetzlich ist das usepackage{beamerthemeshadow} eingebunden 
% \usepackage{beamerthemeshadow}

%  \beamersetuncovermixins{\opaqueness<1>{25}}{\opaqueness<2->{15}}
%  sorgt dafuer das die Elemente die erst noch (zukuenftig) kommen 
%  nur schwach angedeutet erscheinen 
%\beamersetuncovermixins{\opaqueness<1>{25}}{\opaqueness<2->{15}}
% klappt auch bei Tabellen, wenn teTeX verwendet wird\ldots
\begin{document}


\begin{frame}
\titlepage
\end{frame} 

\begin{frame}
\frametitle{Inhaltsverzeichnis}
\tableofcontents
\end{frame} 


\section{Allgemeine Punkte} 
\begin{frame}
\frametitle{Code} 
\begin{itemize}
\item veröffentlichbarer Code
\item lesbarer Code
\item wartbarer Code
\item Trennung von Technik und Fachlichkeit
\end{itemize}
\begin{block}{Ziel}
  Grundsätzlich soll Code erzeugt werden der auf github veröffnetlicht werden kann (nicht muss)
  \end{block}
\end{frame}
\begin{frame}
\frametitle{Sprache} 
\begin{itemize}
\item Entwickliung in Englisch
\item Kommentare in Englisch
\item Dokumentation ausserhalb in deutsch, englisch oder mandarin
\end{itemize}
\begin{block}{Ziel}
  Form follows function - Pragmatische Auswahl der Sprache nach Bedarf, Anwendung und Ziel
  \end{block}
\end{frame}
\begin{frame}
    \frametitle{Steile Thesen} 
    \begin{itemize}
    \item Jede Iteration im Produkt muss das Produkt besser machen
    \item Wenn Excel Teil deines Deployments ist solltest du deine Berufswahl ueberdenken
    \item Wenn die Kommandozeile / Rumgefrickel Teil deines Deployments ist solltest du dein Deployment ueberdenken
    \item Wenn dein Companero deine Vorgehensweise nicht versteht solltest du deine Vorgehensweise ueberdenken
    \item Wenn dein Abnehmer dein Produkt nicht versteht liegt es nicht daran dass dieser kein Informatiker ist.
    \item git clone sollte ausreichen um eine luaffaehige Entwicklungsumgebung zu haben (vorrausgesetyt du sitzt an einem Standardentwicklungsrechner)
    \end{itemize}
    
    \end{frame}
\section{Codestyle}
\begin{frame}
  \frametitle{Formatter}
  \begin{itemize}
  \item Kein Eigenlösungen
  \item google codestyle als Basis
  \item ausschliesslich google formatter verwenden
  \end{itemize}
  \begin{block}{Prämisse}
    Jeder kann machen was er will .... Der Main branch muss konform zum google code style sein
    \end{block}
\end{frame}
%\section{Modulstruktur}
\begin{frame}
    \frametitle{Nomenklatur}
    \begin{itemize}
    \item artifactId == Name des Moduls
    \item groupdid == de.Firmenname.bereichsname.projekt
      \item package name == groupdid.artifactid:Versionsnummer
    \end{itemize}
    \begin{block}{Ziel}
      Einheitliche Namensstrukturen, 
      \end{block}
  \end{frame}
  \begin{frame}
    \frametitle{Generelle Punkte}
    \begin{itemize}
    \item Multimodul vor Multiprojekt
    \item parent pom mit wählbaren plugins
    \item Querabhängigkeiten nie über snapshots
    \end{itemize}
    \begin{block}{Ziel}
      Keine Works on my machine Projekte - Einheitliche Strukturen. Einheitliche Arbeitsumgebung (TODO besseres Wortwahl)
      \end{block}
  \end{frame}
\section{Workflow}
\begin{frame}
    \frametitle{Branches}
    \begin{itemize}
    \item Main-branch heisst main
    \item Entwicklungsbranches sind (kurze) featurebranches
    \item Merge in main erfolgt immer ueber pull-request
    \item In der Regel sollte ein Code-review stattfinden
    \item Featurebranchname: $<JIRA-TicketNo>\_<sprechende\_Bezeichnung>$
    \end{itemize}
    \begin{block}{Ziel}
      Im main-branch sollte immer lauffaehiger, getesteter und deployfahiger Code liegen
      \end{block}
  \end{frame}

\begin{frame}
    \frametitle{git}
    \begin{itemize}
    \item squash commits bei merge
    \item Grundsaetzlich nur fast-forward merges
    \item Ansonsten: $ main  \rightarrow feature \rightarrow main $
    \item rebase $\rightarrow$ Gehe in das Gefängnis. Begib Dich direkt dorthin. Gehe nicht über Los. Ziehe nicht EUR 1000 ein 
    \end{itemize}
    \begin{block}{Ziel}
      Uebersichtliche und nachvollziehbare Entwicklungszyklen sowie git als single point of truth
      \end{block}
\end{frame}

\begin{frame}
    \frametitle{CI}
    \begin{itemize}
      \item Verwendung der Basis-CI
      \item Kein voodoo in der CI (dwyce - do what your customer expect)
      \item Feature-branch pushes sollen KEINE Artefakte o.ae. erzeugen
      \item Ein push auf main muss ein Artefakt erzeugen (release)
    \end{itemize}
    \begin{block}{Ziel}
        Die CI stellt sicher dass getesteter und lauffaehiger Code im repositry liegt. Zudem stellt Sie sicher dass Artefakte luaffaehig und versioniert abgelegt werden
    \end{block}
\end{frame}

\begin{frame}
    \frametitle{CI 2}
    \begin{itemize}
      \item Artifakte sollten nur als release abgelegt werden
      \item CI fuehrt tests durch z.B. spotless:check, pmd:check, sonarcube, jacoco \dots
      \item Builds sollen reproduzierbar sein
    \end{itemize}
    \begin{block}{Ziel}
        Einige Metriken der CI sollen auswertbar sein und Codeveraenderungen aussagekraeftig anzeigen koennen.
    \end{block}
\end{frame}


\section{Repositories / Registries}
\begin{frame}
    \frametitle{Git Repository / gitlab}
    \begin{itemize}
      \item Multimodul vor Multiprojekt
      \item Branches muessen jederzeit baubar sein
      \item Ein Entwickler sollte keine Anpassungen vornehmen muessen um einen Code bauen zu koennen
      \item Voodoo sollte vermieden werden
    \end{itemize}
    \begin{block}{Ziel}
        Das Gitlab sollte jederzeit in einem Zustand sein sodass es fuer die Allgemeinheit als open source zur Verfuegung gestellt werden koennte. (Rein qualitativ betrachtet) 
    \end{block}
\end{frame}
\begin{frame}
    \frametitle{Nomenklatur}
    \begin{itemize}
    \item artifactId == Name des Moduls
    \item groupdid == de.Firmenname.bereichsname.projekt
      \item package name == groupdid.artifactid:Versionsnummer
    \end{itemize}
    \begin{block}{Ziel}
      Einheitliche Namensstrukturen, sprechende Namen und Nachvollziehbarkeit
      \end{block}
\end{frame}
\section{Rollen / Berechtigungen}
\begin{frame}
    \frametitle{Projektrollen}
    \begin{itemize}
    \item group ownership sollten nur sehr wenige benutzen muessen
    \item Standardrolle im eigenen Projekt als Projektverantwortlicher sollte Maintainer sein
    \item Standardrolle fuer Vertreter sollte Maintainer sein
    \item Alle anderen sollten Entwickler sein
    \item Der eigene Space kann fuer Spielprojekte mitt eigenem Ownership genutzt werden
    \end{itemize}
    \begin{block}{Ziel}
        Einige Wenige sind verantwortlich fuer die Struktur des gesamten git. Fuer die Codequalitaet eines Projektes ist der Maintainer verantwortlich.
    \end{block}
\end{frame}
\begin{frame}
    \frametitle{Projektrollen 2}
    \begin{itemize}
    \item Owner ist zustaendig fuer die Gruppe
    \item Mainainer sind verantwortlich dafuer dass der main branch laeuft
    \item Aendrungen werden nur durch pull requests durchgefuehrt
    \item Maintainer fuehren die Code Reviews durch (aber nicht von eigenem Code)
    \end{itemize}
    \begin{block}{Ziel}
        Es geht darum qualitativ hochwertige Software zu erstellen \dots Pragmatismus geht jedoch vor Buerokratie \dots
    \end{block}
\end{frame}



\end{document}
